\begin{titlepage}
    \begin{center}
        \vspace*{1cm}
        
        \includegraphics[width=0.4\linewidth]{figuras/cuenca.png}\\
        \LARGE
        \unidad\\
        \programa\\
        \curso
        
        \vspace{1cm}
        
        \Huge
        \textbf{\titulo}
            
        \vspace{0.5cm}
        \LARGE
        \subtitulo
            
        \vspace{1.5cm}
        
        \large    
        \autores
            
        \vfill
        
        \lugar\\
        \fecha
        
    \end{center}
\end{titlepage}


\documentclass[journal]{IEEEtran}
\usepackage[utf8]{inputenc}
\usepackage[english]{babel}
\usepackage{hyperref}
\usepackage{apacite}
\usepackage{graphicx}
\usepackage{listings}
\usepackage{siunitx}
%====================================================%

\lstset{
    language=C,
    basicstyle=\footnotesize,
    numbers=left,
    stepnumber=1,
    showstringspaces=false,
    tabsize=1,
    breaklines=true,
    breakatwhitespace=false,
}

%====================================================%
\def \unidad{Universidad de Cuenca}
\def \programa{Ingeniería en Telecomunicaciones}
\def \curso{Grupo \# 3 - Cálculo en Varias Variables}
\def \titulo{Trabajo en clase}
\def \subtitulo {Resolución de Ejercicios}
\def \autores{

    
    Javier Alexander Procel Morocho\\
    javier.procel@ucuenca.edu.ec\\

    \vspace{0.5cm}
    
    Julio Josué Macas Guamán\\
    julio.macas@ucuenca.edu.ec\\
    
    \vspace{0.5cm}
    Christian Eduardo Mendieta Tenesaca\\
    eduardo.mendieta@ucuenca.edu.ec\\

    \vspace{0.5cm}
    Paul Alejandro Minga Fajardo\\
    paul.minga@ucuenca.edu.ec\\
}
\def \fecha{Octubre 2023}
\def \lugar{
    Azuay, 
    Cuenca
}

%====================================================%
\begin{document}
\input{portada}
\tableofcontents \vspace{2mm}

%====================================================%
\begin{abstract}
This report will present the resolution of the exercises proposed in the class, which will help us understand certain concepts of differential calculus in several variables.
\end{abstract}

%====================================================%
\renewcommand{\IEEEkeywordsname}{Keywords}
\begin{IEEEkeywords}
derivada, variación, tiempo, diferencial
\end{IEEEkeywords}

%====================================================%
\section{Ejercicio 14.3 (83)}\label{intro}

La resistencia total R producida por tres conductores con resistencias \(R_{1}, R_{2}, R_{3}\) conectadas en un circuito eléctrico paralelo está dada por la fórmula:

\[\frac{1}{R} = \frac{1}{R_{1}} + \frac{1}{R_{2}} + \frac{1}{R_{3}}\]

Determine \(\partial R / \partial R_{1}\).
\vspace{0.5cm}
\begin{enumerate}
    \item[(a)] Despejamos \(R\):

        \[\frac{1}{R} = \frac{1}{R_1} + \frac{1}{R_2} + \frac{1}{R_3}\]
        
        \[1 = \left(\frac{1}{R_1} + \frac{1}{R_2} + \frac{1}{R_3}\right) \times R\]
        
        \[R = \frac{1}{\frac{R_2R_3 + R_1R_3 + R_1R_2}{R_1R_2R_3}}\]
        
        \[R = \frac{R_1R_2R_3}{R_2R_3 + R_1R_3 + R_1R_2}\]

    \item[(b)] Derivamos el numerador y el denominador con respecto a  \(R_{1}\):

        \[ \frac{\partial (R_1R_2R_3)}{\partial R_{1}} = R_2R_3 \]
        \[ \frac{\partial (R_2R_3 + R_1R_3 + R_1R_2)}{\partial R_{1}} =R_3 +  R_2 \]

    \item[(c)] Determinamos \(\partial R / \partial R_{1}\):

        \[\frac{\partial R}{\partial R_{1}}\]
        \[
             = \frac{(R_2R_3 + R_1R_3 + R_1R_2)(R_2R_3) - (R_1R_2R_3)(R_3 +  R_2)}{(R_2R_3 + R_1R_3 + R_1R_2)^{2}}
        \]

        \[ = \frac{R_2^{2}R_3^{2} + R_1R_2R_3^{2} + R_1R_2^{2}R_3 - R_1R_2R_3^{2} - R_1R_2^{2}R_3}{(R_2R_3 + R_1R_3 + R_1R_2)^{2}}\]
        \[ = \frac{R_2^{2}R_3^{2}}{(R_2R_3 + R_1R_3 + R_1R_2)^{2}}\]

\end{enumerate}
%====================================================%
\section{Ejercicio 14.3 (100)}

En un estudio de penetración de la escarcha se descubrió
que la temperatura T en el tiempo t (medido en días) a una profundidad x (medida en metros) puede modelarse con la función

\[T(x, t) = T_{o} + T_{1e}^{-\lambda x} sen(\omega t-\lambda x) \]

donde \(\omega = 2\pi / 365\) y \(\lambda \) es una constante positiva.

\begin{enumerate}
    \item[(a)] Encuentre \(\partial T / \partial x\). ¿Cuál es su significado físico?
    
    \[\]
    El significado físico de \(\partial T / \partial x\) es la tasa de cambio de la temperatura con respecto a la profundidad x. Es decir, indica cómo la temperatura cambia a medida que nos movemos en la dirección de la profundidad.

    \

    \item[(b)] Encuentre \(\partial T / \partial t\). ¿Cuál es su significado físico?
    
    El significado físico de \(\partial T / \partial t\) es la tasa de cambio de la temperatura con respecto al tiempo t. Indica cómo la temperatura cambia a medida que transcurre el tiempo.

    \
    
    \item[(c)] Demuestre que T satisface la ecuación de calor \(T_{t} = kT_{xx}\) para cierta constante k.
    
    
    
    \item[(d)] Si \(\lambda = 0.2\), \(T_{o} = 0\) y \(T_{1} = 10\) use una computadora para graficar T(x, t).
    
    \begin{figure}[h]
    \centering
    \includegraphics[width=0.2\textwidth]{figuras/c1.png}
    \caption{Vista lateral}
    \label{fig:enter-label}
    \end{figure}

    \begin{figure}[h]
    \centering
    \includegraphics[width=0.2\textwidth]{figuras/c2.png}
    \caption{Vista normal}
    \label{fig:enter-label}
    \end{figure}
    
    \item[(e)] ¿Cuál es el significado físico del término \(-\lambda x \) en la expresión \(sen(\omega t - \lambda x)\)?
    
    El termino \(-\lambda x \) en la expresión \(sen(\omega t - \lambda x)\) representa una fase espacial. Indica cómo la onda sinusoidal (que modela la variación de temperatura) cambia a lo largo de la dirección de la profundidad (x). En otras palabras, \(-\lambda x \) determina la posición inicial de la onda sinusoidal en el eje x. A medida que x aumenta, la fase espacial cambia, lo que afecta la distribución de temperatura a diferentes profundidades.
    
\end{enumerate} \newpage
%====================================================%
\section{Ejercicio 14.4 (5)}

    Determine una ecuación del plano tangente a la superficie 
    dada en el punto especificado.

    \[z = \sqrt{xy}, \quad (1, 1, 1)\]
    \[ f(x, y) = \sqrt{xy}\]
    \[f_x(x, y) = \frac{1}{2\sqrt{x}} = \frac{1}{2}\]
    \[f_y(x, y) = \frac{1}{2\sqrt{y}} = \frac{1}{2}\]
    \[z - 1 = \frac{1}{2}(x - 1) + \frac{1}{2}(y - 1)\]
    \[z = \frac{1}{2}x - \frac{1}{2} + \frac{1}{2}y - \frac{1}{2} + 1\]
    \[z = \frac{1}{2}x + \frac{1}{2}y\]
    
   
%====================================================%
\section{Ejercicio 14.4 (34)}

    Use diferenciales para estimar la cantidad de metal en una lata cilíndrica cerrada de 10 cm de alto y 4 cm de diámetro  si el metal en la tapa y el fondo es de 0.1 cm de grosor y el  metal en los lados es de 0.05 cm de grosor.

    \[V = \pi r^2 h\]
    \[\partial V = \left(\frac{\partial V}{\partial r}\right)dr + \left(\frac{\partial V}{\partial h}\right)dh\]
    \[= 2\pi rh\,dr + \pi r^2\,dh\]
    \[\partial V = 2\pi(2)(10)(0.1) + \pi(2^2)(0.05)\]
    \[\partial V = \left(\frac{21}{5}\right)\pi\]
    \[\partial V = 13.19 cm^3\]
    
%====================================================%
\section{Ejercicio 14.5 (2)}

2) Determinar dz/dt o dw/dt. \vspace{2mm}

\[z = \frac{x-y}{x+2y}\hspace{2mm} , \hspace{2mm} x = e^{\pi t}\hspace{2mm} , \hspace{2mm} y = e^{-\pi t}\]

 \[\frac{d z}{dt} = \frac{\partial z}{\partial x} (\frac{d x}{d t}) + \frac{\partial z}{\partial y}  (\frac{d y}{d t})\]

\[\frac{d z}{dt} = \frac{1(x+2y)-(x-y)(1)}{(x+2y)^2} (\pi e^{\pi t}) + \frac{(-1)(x+2y)-(x-y)(2)}{(x+2y)^2}  (-\pi e^{-\pi t})\]

\[\frac{d z}{dt} = \frac{x+2y-x+y}{(x+2y)^2} (\pi e^{\pi t}) + \frac{-x-2y-2x+2y}{(x+2y)^2}  (-\pi e^{-\pi t})\]

\[\frac{d z}{dt} = \frac{3y\pi e^{\pi t}} {(x+2y)^2} + \frac{3x\pi e^{-\pi t}} {(x+2y)^2}  \]



%====================================================%

\section{Ejercicio 14.5 (40)}

40) El voltaje V en un circuito eléctrico simple disminuye
lentamente conforme se agota la batería. La resistencia R
se reduce con lentitud conforme el resistor se calienta.
Use la ley de Ohm, V = IR, para determinar cómo cambia
la corriente I en el momento en que R = 400 $\si{\ohm}$,
I = 0.08 A, dV/dt = -0.01 V/s, y dR/dt = 0.03 $\si{\ohm}$ /s


\[I = \frac{V}{R}\]

\[\frac{d I}{dt} = \frac{\partial I}{\partial V} (\frac{d V}{d t}) + \frac{\partial I}{\partial R}  (\frac{d R}{d t})\]

\[\frac{d I}{dt} = \frac{1}{R} (-0.01) - \frac{V}{R^2}  (0.03)\]

\[\frac{d I}{dt} = \frac{1}{R} (-0.01) - \frac{I(R)}{R^2}  (0.03)\]

\[\frac{d I}{dt} = \frac{1}{400} (-0.01) - \frac{0.08}{400}  (0.03)\]

\[\frac{d I}{dt} = -0.000031 \hspace{2mm}\frac{A}{s}\]
%====================================================%
\end{document}


