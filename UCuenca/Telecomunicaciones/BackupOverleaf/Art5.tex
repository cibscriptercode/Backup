\begin{titlepage}
    \begin{center}
        \vspace*{1cm}
        
        \includegraphics[width=0.4\linewidth]{figuras/cuenca.png}\\
        \LARGE
        \unidad\\
        \programa\\
        \curso
        
        \vspace{1cm}
        
        \Huge
        \textbf{\titulo}
            
        \vspace{0.5cm}
        \LARGE
        \subtitulo
            
        \vspace{1.5cm}
        
        \large    
        \autores
            
        \vfill
        
        \lugar\\
        \fecha
        
    \end{center}
\end{titlepage}




\documentclass[journal]{IEEEtran}
\usepackage[utf8]{inputenc}
\usepackage[spanish]{babel}
\usepackage{hyperref}
\usepackage{apacite}
\usepackage{graphicx}
\usepackage{listings}
\usepackage{amsmath}
%====================================================%

\lstset{
    language=C,
    basicstyle=\footnotesize,
    numbers=left,
    stepnumber=1,
    showstringspaces=false,
    tabsize=1,
    breaklines=true,
    breakatwhitespace=false,
}

%====================================================%
\def \unidad{Universidad de Cuenca}
\def \programa{Ingeniería en Telecomunicaciones}
\def \curso{Grupo \# 3 - Cálculo de Varias Variables}
\def \titulo{Tarea en Clase 3.3}
\def \subtitulo {Investigación}
\def \autores{
    Javier Alexander Procel Morocho\\
    javier.procel@ucuenca.edu.ec\\
    
    \vspace{0.5cm}
    
    Paul Alejandro Minga Fajardo\\
    paul.minga@ucuenca.edu.ec\\

    \vspace{0.5cm}
    
    Julio Josué Macas Guamán\\
    julio.macas@ucuenca.edu.ec\\

    \vspace{0.5cm}
    
    Christian Eduardo Mendieta Tenesaca\\
    eduardo.mendieta@ucuenca.edu.ec\\
}
\def \fecha{Noviembre 2023}
\def \lugar{
    Azuay, 
    Cuenca
}

%====================================================%
\begin{document}
\input{portada}
\tableofcontents

%====================================================%
\begin{abstract}
This report talks about very important topics for calculation and optimization. It begins with the topic of directional derivatives where we investigated how functions change in specific directions. Through an example it was possible to understand what was investigated in a better way.\\
As a second topic, the topic of maximums and minimums was explored, where the critical points and the use of derivatives were analyzed to determine the nature of these local extremes. The report presents the procedure on how to find these values step by step, followed by a couple of very detailed examples.\\
Following that topic, the topic of the Taylor polynomial was analyzed, where it serves as a great mathematical tool. This topic presents its formulation and illustrates its application in the approximation of complex functions using simpler polynomials.\\
Finally, the topic of Lagrange multipliers was addressed. This topic explores how this methodology allows finding conditional extremes, providing an effective solution for optimization problems in which multiple variables coexist.
\end{abstract}

%====================================================%
\renewcommand{\IEEEkeywordsname}{Palabras Clave}
\begin{IEEEkeywords}
Directional derivatives, gradient vector, maxima and minima, critical points, Taylor polynomial, Lagrange multipliers, second derivative criterion and polynomial approximation
\end{IEEEkeywords}

%====================================================%
\section{Introducción}\label{intro}

    Las derivadas direccionales constituyen una herramienta fundamental en el cálculo vectorial, permitiendo medir la tasa de cambio de una función en una dirección específica. Esta técnica se expresa mediante notaciones como \(f_{\mathbf{v}}\), \(\mathbf{D_vf}\), y \(\frac{\partial f}{\partial \mathbf{v}}\), ofreciendo una perspectiva valiosa sobre cómo varía una función en un punto dado al desplazarse en una dirección particular. Posteriormente, se explora el concepto de gradientes, indicando la dirección y magnitud máxima de cambio de una función. A través de ejemplos concretos, se demuestra cómo calcular derivadas direccionales y gradientes en situaciones prácticas.

    El análisis de máximos y mínimos en funciones de varias variables implica técnicas específicas, desde encontrar derivadas parciales hasta aplicar criterios de segunda derivada para determinar la naturaleza de los puntos críticos. Se ilustra este proceso mediante ejemplos detallados, subrayando la importancia del criterio de la segunda derivada para clasificar puntos críticos como máximos, mínimos o puntos de silla. Además, se presenta el polinomio de Taylor como una herramienta esencial en el cálculo y análisis matemático, ofreciendo aproximaciones locales precisas de funciones suaves. Finalmente, se explora el método de multiplicadores de Lagrange, diseñado para encontrar máximos o mínimos de funciones sujetas a restricciones, presentando un ejemplo concreto para ilustrar su aplicación. Estas herramientas matemáticas son esenciales para comprender y modelar fenómenos en diversos campos científicos y aplicados.

%====================================================%
\section{Derivadas direccionales}
Una derivada direccional es un método que se proporciona para medir la tasa de cambio que existe en una direcciona dada, es decir, indica el valor de una función en un punto dado cuando este se desplaza a una dirección específica.\cite{Redondo2012} \\
Para este concepto, existen varias notaciones, como por ejemplo: 
\begin{equation}
    f´\vec{v}
\end{equation}
\begin{equation}
    D\vec{v}f
\end{equation} 
\begin{equation}
    \frac{\partial f}{\partial \vec{v}}
\end{equation}
Para poder resolver una derivada direccional se emplea la siguiente fórmula. Ahora, supongamos que tenemos una función f(x, y) en un punto P (xo, yo). Entonces, la derivada direccional de f en el punto P dado, en una direcciona del vector unitario v = (a, b), esta denotado por Dvf(P) (Dvf(xo,yo)) y se calcula de la siguiente manera:
\begin{equation}
    Dvf(P) = \lim_{h\rightarrow 0} \frac{f(x_o + ha, y_o + hb) - f(x_o, y_o)}{h}
\end{equation}

\subsection{Vector gradiente}
El gradiente indica la dirección y la magnitud máxima de la tasa de cambio de una función en un punto dado. La función del vector está definida por: \cite{Redondo2012}
\begin{equation}
    \bigtriangledown f(x, y) = (f_x(x, y), f_y(x, y)) = \frac{\partial f}{\partial x}i + \frac{\partial f}{\partial y}j
\end{equation}
Cada componente del gradiente indica la tasa de cambio de f con respecto a la correspondiente variable. Con respecto a la derivada direccional de f en el punto P en direccion del vector unitario u, está definido por:
\begin{equation}
    D_uf(x, y) = \bigtriangledown f(x, y) * u
\end{equation}

\subsection{Ejemplo:}
Encontrar la derivada direccional de la funcion f(x, y) = x2 - 4xy en el punto P(1, 2) y el vector i + 3j\\
Entonces empezamos calculando el gradiente
\begin{equation}
    \bigtriangledown f = (2x - 4y)i - (4x)j
\end{equation}
Seguido remplazamos el punto que nos dan en el ejercicio
\begin{equation}
    \bigtriangledown f(1, 2) = (2(1) - 4(2))i - (4(1))j
\end{equation}
\begin{equation}
    \bigtriangledown f(1, 2) = -6i - 4j
\end{equation}
Encontramos nuestro vector unitario
\begin{equation}
    u = \sqrt{1^2 + 3^2}
\end{equation}
\begin{equation}
    u = \frac{i + 3j}{\sqrt{10}}
\end{equation}
\begin{equation}
    u = \frac{1}{\sqrt{10}}i + \frac{3}{\sqrt{10}}j
\end{equation}
Encontramos la derivada direccional utilizando la formula de la ecuacion 6
\begin{equation}
    D_uf(x, y) = (-6i - 4j)*(\frac{1}{\sqrt{10}}i + \frac{3}{\sqrt{10}}j)
\end{equation}
\begin{equation}
    D_uf(x, y) = -1.89 - 3.79
\end{equation}
\begin{equation}
    D_uf(x, y) = -5.68
\end{equation}


%====================================================%
\section{Valores máximos y mínimos}
    Los máximos y mínimos en una función de varias variables se refieren a los puntos donde la función alcanza los valores más altos (máximos) o más bajos (mínimos). 

    \subsection{Procedimiento:}
        \begin{enumerate}
            \item Encontrar Derivadas Parciales: 
            Calcular la primera derivada parcial con respecto a \textit{x} (denotada como $\frac{\partial f}{\partial x}$) y la primera derivada parcial con respecto a \textit{y} (denotada como $\frac{\partial f}{\partial y}$).

            
            Estas derivadas parciales proporcionan información sobre la tasa de cambio de la función en cada dirección. \vspace{0.1cm}
            
            \item Encontrar Puntos Críticos: 
            Establecer $\frac{\partial f}{\partial x}=0$ y$\frac{\partial f}{\partial y}=0$ para encontrar los puntos críticos.
            
            Resolver el sistema de ecuaciones resultante para determinar los valores de \textit{x} e \textit{y} en los cuales las derivadas parciales son cero.
            \vspace{0.1cm}
            
            \item Aplicar Criterio de la Segunda Derivada: 
            Calcular las segundas derivadas parciales, como \textit{fxx}, \textit{fyy} y \textit{fxy}.
            Utilizar el criterio de la segunda derivada para determinar la naturaleza de los puntos críticos.
            
            Si $fxxfyy - (fxy)^2$ es positivo y \textit{fxx} es positivo, el punto es un mínimo; si es negativo, es un máximo. Si $fxxfyy - (fxy)^2$ es igual a cero, el criterio no concluye y se necesita otra prueba.
            \vspace{0.1cm}
            
            \item Identificar Máximos y Mínimos: 
            Evaluar la función en los puntos críticos y comparar los valores para determinar si son máximos o mínimos.
            En el caso de un punto crítico, si la segunda derivada es positiva, es un mínimo; si es negativa, es un máximo.
            \vspace{0.1cm}
            
            \item Verificar con Derivadas Cruzadas: 
            Comprobar los resultados utilizando las derivadas cruzadas ($\frac{\partial^2 f}{\partial x \partial y}$ y $\frac{\partial^2 f}{\partial y \partial x}$) en el criterio de la segunda derivada. 
            \vspace{0.1cm}
        \end{enumerate}
        
    \subsection{Ejemplo 1:} \cite{matefacil2020}
    

            Dada la función $f(x, y) = x^2 + y^2 - xy$ \vspace{0.2cm}

            1. Derivadas parciales:
                \begin{align*}
                    f_x &= 2x - y \\
                    f_y &= 2y - x
                \end{align*}
            
            2. Igualando a cero:
                \begin{align*}
                    2x - y &= 0 \quad \Rightarrow \quad 2x = y \\
                    2y - x &= 0 \quad \Rightarrow \quad 3x = 0 \quad \Rightarrow \quad x = 0 \\
                    &\quad \Rightarrow \quad y = 0
                \end{align*} 
                Un punto crítico en $(0, 0)$. \vspace{0.2cm}
            
            3. Segundas derivadas:
                \begin{align*}
                    f_{xx} &= 2, \quad f_{yy} = 2, \quad f_{xy} = -1
                \end{align*}
            
           4. Calcular el determinante para cada punto crítico:
                \begin{align*}
                    D &= (f_{xx})(f_{yy}) - (f_{xy})^2 \\
                    D &= (2)(2) - (-1)^2 = 4 - 1 = 3
                \end{align*}

                \vspace{0.2cm}
                Conclusión:
                \vspace{0.2cm}
                
                \[ f(x, y) = x^2 + y^2 - xy \]
                \[ f(0, 0) = 0^2 + 0^2 - 0 \cdot 0 \]
                \[ D > 0 \quad \text{y} \quad f_{xx} > 0 \quad \text{entonces hay un mínimo en el punto} \ (0, 0) \]
                \vspace{0.2cm}
                
    \subsection{Ejemplo 2:} \cite{matefacil2020}

        Dada la función $f(x, y) = e^{2 + x^2 - y^2}$  \vspace{0.2cm}

        1. Derivadas parciales:
        \begin{align*}
            f_x &= e^{2 + x^2 - y^2} \cdot 2x \\
            f_y &= e^{2 + x^2 - y^2} \cdot (-2y)
        \end{align*}
        
        2. Igualando a cero:
        \begin{align*}
            e^{2 + x^2 - y^2} \cdot 2x &= 0 \quad \Rightarrow \quad x = 0 \\
            e^{2 + x^2 - y^2} \cdot (-2y) &= 0 \quad \Rightarrow \quad y = 0
        \end{align*}
        Un punto crítico en $(0, 0)$.  \vspace{0.2cm}
        
        3. Segundas derivadas:
        \begin{align*}
            f_{xx} &= e^{2 + x^2 - y^2} \cdot (2x)^2 + e^{2 + x^2 - y^2} \cdot 2 \\
            f_{yy} &= e^{2 + x^2 - y^2} \cdot (-2y)^2 + e^{2 + x^2 - y^2} \cdot (-2) \\
            f_{xy} &= e^{2 + x^2 - y^2} \cdot (-2y) \cdot (2x)
        \end{align*}
        
        Evaluando en $(0, 0)$:
        \begin{align*}
            f_{xx}(0, 0) &= 2e^2 \\
            f_{yy}(0, 0) &= -2e^2 \\
            f_{xy}(0, 0) &= 0
        \end{align*}
        
        4. Calcular el determinante para cada punto crítico:
        \begin{align*}
            D &= (f_{xx})(f_{yy}) - (f_{xy})^2 \\
            D &= (2e^2)(-2e^2) - (0)^2 = -4e^4
        \end{align*}
        
        Conclusión:
        
        \[ D < 0, \quad \text{hay un punto de silla en el punto} \ (0, 0) \]
         \vspace{0.2cm}
%====================================================%
\section{Polinomio de Taylor}
Los polinomios de Taylor son una herramienta fundamental en el campo del cálculo y el análisis matemático. Nombrados en honor al matemático británico Brook Taylor, estos polinomios proporcionan aproximaciones locales de funciones suaves mediante polinomios.

\

La aproximación por medio de una recta tangente L(x) es la mejor aproximación de primer grado (lineal) a f(x) cerca de x = a porque f(x) y L(x) tienen la misma razón de cambio (derivada) en a. Para tener una mejor aproximación que una lineal, intente una aproximación de segundo grado (cuadrática) P(x). En otras palabras, aproxime una curva mediante una parábola, en lugar de utilizar una recta. 

Para tener la seguridad de que la aproximación es buena, 
establezca lo siguiente:
\begin{itemize}
    \item[(i)] \(P(a) = f(a)\) (P y f deben tener el mismo valor en a.)
    \item[(ii)] \(P'(a) = f'(a)\) (P y f deben tener la misma razón de cambio en a.)
    \item[(iii)] \(P''(a) = f''(a)\) (Las pendientes de P y f deben tener la misma razón de cambio en a.)
\end{itemize}

 Polinomio de Taylor de grado \(n\) para una función \(f(x)\) centrado en \(a\):
 \(T_n(x) = f(a) + f'(a)(x-a) + \frac{1}{2!}f''(a)(x-a)^2 + \frac{1}{3!}f'''(a)(x-a)^3 + \ldots + \frac{1}{n!}f^{(n)}(a)(x-a)^n\) \vspace{2mm}

 Polinomio de Taylor de varias variables de grado k para f centrado en a: \vspace{2mm}

 $T_k(f, \mathbf{x}; \mathbf{a}) = f(\mathbf{a}) + \sum_{|\mathbf{\alpha}|=1}^{k} \frac{\mathbf{D}^\mathbf{\alpha}f(\mathbf{a})}{\mathbf{\alpha}!} (\mathbf{x} - \mathbf{a})^\mathbf{\alpha}
$ \vspace{2mm}

 Los coeficientes del polinomio de Taylor están relacionados con las derivadas de la función \(f(x)\) evaluadas en el punto \(a\). El término \(\frac{1}{n!}f^{(n)}(a)(x-a)^n
\) representa la contribución de la n-ésima derivada en la aproximación. Cuanto mayor sea n, mayor será la precisión de la aproximación.
 

%====================================================%
\section{Multiplicadores de Lagrange}

Este método fue creado por Jose Luis Lagrange, con el propósito de poder encontrar los valores máximos o mínimos de una función de la forma $f(x,y,...)$ que tiene una restricción de la forma $g(x,y,...) = c$.

Para esto, se agrega un valor $\lambda $ a la igualdad de gradientes de ambas funciones, quedando dos ecuaciones, las cuáles se resolverán de forma simultanea:\vspace{2mm}

\centering
$\bigtriangledown f(x,y,...) = \lambda \cdot  \bigtriangledown g(x,y,...)$ \vspace{1mm}

$g(x,y,...) = c$ \vspace{3mm}

En el caso de los gradientes, si se quiere expresar en forma de sus componentes (hasta z), quedan de la siguiente forma:\vspace{2mm}

$f_{x} = \lambda g_{x}$ \hspace{2mm} $f_{y} = \lambda g_{y}$\hspace{2mm} $f_{z} = \lambda g_{z}$ \vspace{2mm}

Siendo  $f_{x,y,z}$ y $g_{x,y,z} $ las derivadas de las funciones iniciales. \vspace{3mm}

Ejemplo: \vspace{3mm}

$f(x,y) = 9-x^2-y^2$ con la condición:  $x+y = 3$ \vspace{1mm}

$f_{x} = \lambda g_{x}$ \vspace{1mm}

$f_{y} = \lambda g_{y}$\vspace{1mm}

$g(x,y) = c$ \vspace{4mm}

\left\{\begin{matrix}
-2x = \lambda (1)\\ 
2y = -\lambda (1)\\ 
 2(x+y-3) = 0\\ 
\end{matrix}\right. \vspace{3mm}


$-2x-+2y+2x+2y-6 = 0 \lambda$ \vspace{2mm}

$4y = 6 $ \vspace{2mm}

$y = \frac{3}{2} $ \vspace{2mm}

$\lambda = -3$ \vspace{2mm}

$x = \frac{3}{2}$ \vspace{2mm}

$f(\frac{3}{2},\frac{3}{2}) = 9-(\frac{3}{2})^2-(\frac{3}{2})^2 = \frac{9}{2}$






%====================================================%
\section{Conclusiones}\label{ejemplos}

\begin{itemize}
  \item Las derivadas direccionales es una aplicación muy útil para el cálculo vectorial, ya que nos permite analizar y comprender como cambia una función en una dirección específica dada con la ayuda de una formula y el vector gradiente que son de mucha ayuda para encontrar la derivada direccional.  
  
  \item Encontrar máximos y mínimos en funciones de varias variables implica aplicar conceptos del cálculo multivariable para identificar puntos críticos y determinar la naturaleza de estos puntos. La derivación parcial y el criterio de la segunda derivada son herramientas fundamentales en este proceso. Al igual que en funciones de una variable, se busca entender si un punto crítico representa un máximo, un mínimo o un punto de silla. 
  
  \item Los polinomios de Taylor ofrecen una herramienta poderosa en matemáticas y ciencias aplicadas. Su capacidad para proporcionar aproximaciones precisas de funciones y modelos locales los hace indispensables en diversos campos. La comprensión de los polinomios de Taylor amplía la perspectiva sobre la naturaleza local y global de las funciones matemáticas, brindando a los investigadores y profesionales herramientas valiosas para el análisis y la modelización.
  
  \item El método de multiplicadores de Lagrange se revela como una herramienta valiosa en la optimización de funciones sujetas a restricciones. Al incorporar factores multiplicativos, este enfoque permite encontrar los puntos estacionarios de la función objetivo bajo las restricciones dadas. La eficacia de este método radica en su capacidad para simplificar problemas de optimización con restricciones, transformándolos en un sistema de ecuaciones que puede resolverse de manera eficiente. A través de ejemplos prácticos, se demuestra la aplicabilidad y utilidad de los multiplicadores de Lagrange en la resolución de problemas del mundo real, ofreciendo una perspectiva valiosa en el análisis y la toma de decisiones en diversos campos.
\end{itemize}


%====================================================%
\bibliographystyle{apacite}
\bibliography{referencias}

%====================================================%
\end{document}