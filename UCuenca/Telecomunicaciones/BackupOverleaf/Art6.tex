\documentclass[journal]{IEEEtran}
\usepackage{graphicx}
\usepackage[spanish]{babel}
\usepackage{hyperref}
\usepackage{apacite}
\usepackage{listings}
\usepackage{float}



%===================================================%
\begin{document}

\title{Propuesta de Proyecto Empresarial - Fase Dual de la carrera de Mecatrónica}

\author{Christian Sanchez, Eduardo Mendieta}

\maketitle

\IEEEpeerreviewmaketitle

%===================================================%
\section{Introducción}

\IEEEPARstart{D}{entro} del plan de estudios de la carrera de Mecatrónica en el Instituto Superior Universitario Tecnológico del Azuay, se establece como requisito la elaboración de un proyecto empresarial durante la fase dual de la formación, como parte integral del ciclo académico. En este contexto, se plantea la creación de un proyecto orientado a la generación de energía eléctrica mediante el uso de microgeneradores disponibles en el mercado, los cuales aprovechan la alta presión del agua en movimiento. El objetivo principal consiste en acumular la energía producida por estos generadores en un medio de almacenamiento adecuado para su posterior aplicación en distintos ámbitos.

%===================================================%
\section{Nombre del proyecto:}

Sistema de generación de energía hidroeléctrica y estación de carga - Análisis de viabilidad y diseño inicial.

%===================================================%
\section{Propuesta:}

En Continental Tire Andina S.A., se ejecutan procesos que requieren suministro de agua a alta presión mediante sistemas de tuberías para diversas funciones. Este fluido hidráulico circula en circuitos cerrados con alta presión. La propuesta es aprovechar la energía cinética de este flujo de agua para generar energía eléctrica mediante múltiples generadores instalados a lo largo de las tuberías. La energía resultante se almacenará en una batería u otro dispositivo adecuado como reserva energética. El propósito es utilizar esta energía para establecer una central eléctrica o estación de carga, capaz de satisfacer las necesidades de dispositivos con diferentes requerimientos de voltaje y amperaje. Aunque priorizaremos la carga de las baterías de los montacargas, esta estación requerirá un control específico, implementado mediante software personalizado.

En la fase inicial del proyecto, realizaremos un análisis exhaustivo de viabilidad y diseño preliminar del sistema de generación hidroeléctrica. Se llevará a cabo una investigación detallada para evaluar las alternativas de generadores hidroeléctricos disponibles en el mercado, considerando sus especificaciones técnicas y su idoneidad para nuestras necesidades. También analizaremos los sistemas de almacenamiento de energía más apropiados para garantizar una gestión eficiente de la energía generada.

Una vez completada la fase de investigación inicial, procederemos al desarrollo de un diseño conceptual del proyecto. Esto incluirá la identificación de la ubicación óptima para instalar los generadores hidroeléctricos a lo largo de las líneas de tuberías industriales, considerando variables como la altura de caída del agua y la presión del flujo.

Junto con el diseño conceptual, elaboraremos un plan detallado de desarrollo del proyecto, que incluirá la estimación de los costos asociados con la ejecución, adquisición de equipos, instalación y mantenimiento. Además, identificaremos los equipos y herramientas necesarios para realizar mediciones y verificaciones iniciales en las diferentes líneas de tuberías, asegurando un enfoque preciso y eficiente en esta fase inicial del proyecto.


\section{Turbina hidráulica: Factores clave para una elección efectiva y rentable}

Al elegir una turbina hidráulica, es crucial considerar los siguientes aspectos:

Caudal de agua disponible: La cantidad de agua disponible en un periodo determinado determinará el tipo de turbina más adecuado para aprovechar ese caudal.

Altura de la caída (o cabeza neta): La diferencia de altura entre el nivel del agua arriba y abajo del punto de captación de la energía hidráulica. Esta altura influirá en qué tipo de turbina será más eficiente en una ubicación específica.

Velocidad de rotación requerida: Dependiendo de la aplicación y los dispositivos mecánicos que la turbina alimentará, se requerirá una velocidad de rotación específica. Algunas turbinas pueden ajustarse para adaptarse a diferentes velocidades de rotación.

Eficiencia y rendimiento: Evaluar cómo la turbina convierte la energía hidráulica en energía mecánica es crucial. La eficiencia y el rendimiento pueden afectar el costo operativo y la rentabilidad a largo plazo del proyecto.

Costo inicial y mantenimiento: Considerar el costo inicial de adquisición e instalación de la turbina, así como los costos de mantenimiento a lo largo de su vida útil, es vital para determinar la viabilidad económica del proyecto.

%===================================================%
\section{Bibliografía complementaria:}

    \begin{itemize}
        \item Pelton (PLT) info. (s/f). PowerSpout. Recuperado el 14 de febrero de 2024, de https://www.powerspout.com/pages/pelton-plt-info-1
        \item Crutto (s/f). Crutto.tech. Recuperado el 14 de febrero de 2024, de https://crutto.tech/index.html
        \item No title. (s/f). Aliexpress.com. Recuperado el 14 de febrero de 2024, de https://es.aliexpress.com/item/1005004077485956.html
        \item  Muñoz Barrios, S. A. (2011). Medición de rendimiento de una turbina axial pequeña para su implementación en un pico-generador hidroeléctrico. https://repositorio.uniandes.edu.co/server/api/core/bitstreams/ \newline
        880f053a-d1ab-4ca0-8b69-df52b8943628/content
        \item Rodríguez-Pérez, A. M., Pulido-Calvo, I., Pereira-Villaseñor, M., \\\& Domínguez-Castro, L. (2017). Evaluación de la instalación de microturbinas en redes hidráulicas a presión. V Jornadas de Ingeniería del Agua, Universidade da Coruña, A Coruña, Spain.
https://geama.org/jia2017/wp-content/uploads/ponencias/posters/mo12.pdf
         \item Mejía, A. E., \\\& Londoño, M. H. (2011). Sistemas de almacenamiento de energía y su aplicación en energías renovables. Scientia et technica, 1(47), 12-16.
https://dialnet.unirioja.es/servlet/articulo?codigo=4517879

    \end{itemize}

\end{document}