\documentclass{article}
\usepackage[utf8]{inputenc}
\usepackage{amsmath}

\title{Conversión de coordenadas geográficas en grados, minutos y segundos a UTM PSAD56}
\author{E.M. 200 Ok}
\date{2023-04-02}

\begin{document}

\maketitle

\section{Introducción}

En este artículo, explicaremos cómo convertir una coordenada geográfica en grados, minutos y segundos a coordenadas UTM PSAD56. La conversión se realiza en los siguientes pasos:

\begin{enumerate}
\item Convertir la latitud y longitud a grados decimales.
\item Calcular la zona UTM en la que se encuentra la coordenada.
\item Calcular la coordenada UTM en metros utilizando las fórmulas de conversión.
\end{enumerate}

\section{Convertir a grados decimales}

Para convertir la latitud y longitud a grados decimales, se utiliza la siguiente fórmula:

\begin{equation}
grados_decimales = grados + (\frac{minutos}{60}) + (\frac{segundos}{3600})
\end{equation}

Por ejemplo, si tenemos la siguiente coordenada en grados, minutos y segundos en Ecuador:
\begin{itemize}
\item Latitud: 00° 30' 00" S
\item Longitud: 80° 00' 00" W
\end{itemize}

Podemos convertirla a grados decimales de la siguiente manera:

\begin{equation}
latitud_dec = -0.5
\end{equation}

\begin{equation}
longitud_dec = -80.0
\end{equation}

\section{Calcular la zona UTM}

El siguiente paso es calcular la zona UTM en la que se encuentra la coordenada. Ecuador se encuentra en la zona UTM 17S. La fórmula para determinar la zona UTM es:

\begin{equation}
zona_utm = \lfloor \frac{longitud_dec + 180}{6} \rfloor + 1
\end{equation}

Donde $\lfloor \rfloor$ es la función piso, que devuelve el mayor entero menor o igual al número dado. En este caso, la longitud es negativa, por lo que sumamos 180 grados para obtener un número positivo.

Para el ejemplo de Ecuador, la zona UTM es:

\begin{equation}
zona_utm = \lfloor \frac{-80.0 + 180}{6} \rfloor + 1 = 17
\end{equation}

\section{Calcular la coordenada UTM}

Una vez que se ha determinado la zona UTM, podemos calcular las coordenadas UTM en metros utilizando las siguientes fórmulas:

\begin{equation}
a = \frac{1}{298.257223563}
\end{equation}

\begin{equation}
e^2 = 0.006694380023
\end{equation}

\begin{equation}
N = \frac{a}{\sqrt{1-e^2\sin^2(\phi)}}
\end{equation}

Donde $\phi$ es la latitud en radianes, que se puede calcular a partir de la latitud en grados decimales:

\begin{equation}
\phi = latitud_dec \cdot \frac{\pi}{180}
\end{equation}

En el caso de Ecuador, tenemos:

\begin{equation}
\phi = -0.5 \cdot \frac{\pi}{180} = -0.008727\ \mathrm{rad}
\end{equation}

Sustituyendo los valores de $a$, $e$, y $\phi$ obtenemos:

\begin{equation}
N = \frac{1}{298.257223563} \cdot \sqrt{1 - 0.006694380023\sin^2(-0.008727)} = 9987901.92\ \mathrm{m}
\end{equation}

A continuación, se calcula el radio de curvatura de la Tierra en el plano meridiano de la coordenada:

\begin{equation}
\rho = \frac{a(1-e^2)}{(1-e^2\sin^2(\phi))^{3/2}}
\end{equation}

En el caso de Ecuador, tenemos:

\begin{equation}
\rho = \frac{1}{298.257223563} \cdot \frac{(1-0.006694380023)}{(1-0.006694380023\sin^2(-0.008727))^{3/2}} = 6367025.14\ \mathrm{m}
\end{equation}

A continuación, se calcula la escala de la proyección UTM en la coordenada:

\begin{equation}
k_0 = 0.9996
\end{equation}

Después, se calcula la longitud central del huso UTM en el que se encuentra la coordenada:

\begin{equation}
\lambda_0 = (zona_utm - 1) \cdot 6 - 180 + 3
\end{equation}

En el caso de Ecuador, tenemos:

\begin{equation}
\lambda_0 = (17 - 1) \cdot 6 - 180 + 3 = -81
\end{equation}

Luego, se calcula la diferencia entre la longitud de la coordenada y la longitud central del huso:

\begin{equation}
\Delta\lambda = longitud_dec - \lambda_0
\end{equation}

En el caso de Ecuador, tenemos:

\begin{equation}
\Delta\lambda = -80.0 - (-81) = 1
\end{equation}

A continuación, se calcula la coordenada UTM este, en metros:

\begin{equation}
E = E_0 + \frac{k_0 N (\Delta\lambda + \frac{\Delta\lambda^3}{6}(1-\tan^2(\phi)+\frac{7}{12}\tan^4(\phi)))}{1 + \frac{N \tan^2(\phi)}{\rho}}
\end{equation}

Donde $E_0$ es la coordenada este en metros del meridiano central del huso UTM, que es 500000 m.

En el caso de Ecuador, tenemos:

\begin{equation}
E = 500000 + \frac{0.9996 \cdot 9987901.92 \cdot (1 + \frac{(1)^3}{6}(1-\tan^2(-0.008727)+\frac{7}{12}\tan^4(-0.008727)))}{1 + \frac{9987901.92 \cdot \tan^2(-0.008727)}{6367025.14}} = 185204.49\mathrm{m}
\end{equation}

Finalmente, se calcula la coordenada UTM norte, en metros:

\begin{equation}
N = N_0 + \frac{k_0 (\Delta\lambda^2 \cdot N \cdot \tan(\phi))}{2} + \frac{k_0 (\Delta\lambda^4 \cdot N \cdot \tan(\phi)^3)}{24}(5 - \tan^2(\phi) + 9 \eta^2 + 4 \eta^4)
\end{equation}

Donde $N_0$ es la coordenada norte en metros del ecuador, que en el hemisferio sur es de 10000000 m.

En el caso de Ecuador, tenemos:

\begin{equation}
\eta = \frac{e^2 \cos^2(\phi)}{1-e^2} = 0.006739496742
\end{equation}

\begin{equation}
N = 10000000 + \frac{0.9996 \cdot (1)^2 \cdot 9987901.92 \cdot \tan(-0.008727)}{2} + \frac{0.9996 \cdot (1)^4 \cdot 9987901.92 \cdot \tan(-0.008727)^3}{24}(5 - \tan^2(-0.008727) + 9 \cdot 0.006739496742 + 4 \cdot 0.006739496742^2) = 995729.73\ \mathrm{m}
\end{equation}

Por lo tanto, la coordenada UTM correspondiente a la coordenada geográfica -0.5\textdegree de latitud y -80.0\textdegree de longitud en Ecuador es $17\ 185204.49\ E,\ 995729.73\ N$ en metros, en el sistema de proyección UTM PSAD56.

\end{document}
